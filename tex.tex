\documentclass{article}
\usepackage{hyperref}
\usepackage[frenchb]{babel}


%https://copainsdavant.linternaute.com/p/christelle-hardy-4784401

%doc equivalent https://perso.ens-rennes.fr/math/people/salim.rostam/files/agreg/compl%C3%A9ments/equations_diophantiennes.pdf

%https://tex.stackexchange.com/questions/63895/click-to-go-to-an-anchored-line
\usepackage{lipsum,hyperref}



%labelling https://tex.stackexchange.com/questions/31299/tikz-position-with-different-anchors

%https://perso.imt-mines-albi.fr/~gaborit/latex/latex-in-french.html
%\usepackage[english,francais]{babel}
%\usepackage[english,french]{babel}

\usepackage[frenchb]{babel}

%https://tex.stackexchange.com/questions/248788/the-very-basics-of-french-accents
%\documentclass[12pt]{amsart}

\usepackage[utf8]{inputenc}

%entiers sets https://texblog.org/2007/08/27/number-sets-prime-natural-integer-rational-real-and-complex-in-latex/
%\usepackage{amsfonts} 
% or 
\usepackage{amssymb}

%Majuscules accentuees https://www.xm1math.net/doculatex/caracteres_speciaux.html
\usepackage{eurosym}
\usepackage{systeme}
\usepackage{enumerate}

%https://borntocode.fr/latex-customisation-de-listes-a-puces/
\usepackage{enumitem}


%numerotation https://www.xm1math.net/doculatex/structure.html
\renewcommand{\thesubsection}{\Roman{subsection}}

%latex a) b) c) headings -> https://tex.stackexchange.com/questions/3177/how-to-change-the-numbering-of-part-chapter-section-to-alphabetical-r 
\renewcommand\thesubsubsection{\alph{subsubsection}}



%https://tex.stackexchange.com/questions/75667/change-colour-on-chapter-section-headings-lyx
\usepackage{xcolor}
\usepackage{sectsty}
\chapterfont{\color{blue}}  % sets colour of chapters
\sectionfont{\color{cyan}}
\partfont{\color{red}} 
\subsectionfont{\color{purple}} 
\subsubsectionfont{\color{blue}} 

%https://borntocode.fr/latex-customisation-de-listes-a-puces/
\definecolor{blueish}{RGB}{51,131,255}




%https://tex.stackexchange.com/questions/137073/writing-mod-in-congruence-problems-without-leading-space
\usepackage{amsmath}
\newcommand{\Mod}[1]{\ (\mathrm{mod}\ #1)}



\begin{document}
%Hello world \LaTeX!

%%%%%%%Sommaire + index
%https://perso.imt-mines-albi.fr/~gaborit/latex/latex-in-french.html
%\renewcommand{\contentsname}{Sommaire}
v = 146

\part{Équations diophantiennes du
  %
  1\ier{}  degré $a \cdot x+b \cdot y=c$. Autres exemples d'équations diophantiennes.}


\textbf{Déf 1}
On appelle équation diophantienne à \textit{n} inconnues, une équation du type $P(Y_{1},....Y_{n})=0$ avec
%https://tex.stackexchange.com/questions/261693/latex-element-of-with-two-strokes-%E2%8B%B9

$P \in \mathbb{Z}[X_{1}...X_{n}]$. On cherche les solutions dans $\mathbb{Z}^{n}$.
\subsection{Équations diophantiennes linéaires}
%systeme d equations https://kogler.wordpress.com/2008/03/21/latex-multiline-equations-systems-and-matrices/
%https://en.wikibooks.org/wiki/LaTeX/Advanced_Mathematics
% $a_{11}  \cdot x_{i} + ... + z a_{1m}  \cdot x_{m} = b_{1}$
%\[\begin{cases} a = 2 \\  c = 3 \\ d = 5 \end{cases}\]

% \systeme{ a = 2 , c = 3, d = 5 }     

%Ia
\subsubsection{Équations diophantiennes du $1^{er}$ degré à 2 inconnues $a \cdot x+b \cdot y=c$ .}

Soit (a,b,c) $\in \mathbb{Z}^3$. On cherche (x,y) $\in \mathbb{Z}^2$  tels que  $a \cdot x + b \cdot y = c$ 
\hypertarget{here2}{\textcolor{green}{(*1)} }
%: https://tex.stackexchange.com/questions/63895/click-to-go-to-an-anchored-line


$\sqrt{x^2+y^2}$

\textcolor{cyan}{Prop 1}
On appelle équation diophantienne à \textit{n} inconnues, une équation du type $P(Y_{1},....Y_{n})=0$Une condition nécessaire et suffisanted'existence d'au moins 1 solution de    \hyperlink{here2}{\textcolor{green}{(*1)}}
est pgcd(a,b) divise c.

\textcolor{cyan}{Théorème de Bezout}

a,b sont 2 entiers. a et b sont premiers entre eux ssi il existe (u,v) $\in \mathbb{Z}^2$ tels que  $a \cdot u + b \cdot v=1$

%\textbf{}
\textcolor{cyan}{Prop 2}
Dans le cas où a et b sont premiers entre eux (breadcrumbs : chapeaux chinois congruence calculatrice HP48), une solution de \hyperlink{here2}{\textcolor{green}{(*1)}} est $(x_0,y_0)=(c \cdot u, c \cdot v)$ avec (u,v) dans le théorème de Bezout.

L'ensemble des solutions de \hyperlink{here2}{\textcolor{green}{(*1)}} est alors S=${(x_0+\lambda \cdot b, y_0-\lambda \cdot b ), \lambda \in \mathbb{Z}}$


\textcolor{cyan}{Méthodes de résolution}

%https://tex.stackexchange.com/questions/129951/enumerate-tag-using-the-alphabet-instead-of-numbers

%https://borntocode.fr/latex-customisation-de-listes-a-puces/
\begin{itemize}[label=\textbullet, font=\LARGE \color{blueish}]
\item trouver $(x_0,y_0)$ par divisions euclidiennes successives
\item méthode des congruences : \textbf{exemple} :   $3 \cdot x + 5 \cdot y=1$
  : on cherche x tel que $3 \cdot x \equiv 1 [5] \Leftrightarrow  x \equiv 2 [5]$. D'où $S={(2 +  5 \cdot \lambda, -1-3 \cdot \lambda ), \lambda \in \mathbb{Z}}$
  %: https://tex.stackexchange.com/questions/137073/writing-mod-in-congruence-problems-without-leading-space
  %mod en crochets 1 \mod 5

  %equivalents https://www.geeksforgeeks.org/equality-and-inference-symbols-in-latex/
  %   \Leftrightarrow https://oeis.org/wiki/List_of_LaTeX_mathematical_symbols
\end{itemize}

%Ib
\subsubsection{Systèmes d'équations diophantiennes linéaires}
Soit (m,n) $\in \mathbb{Z}^2$,$(a_{11},...,a_{1m},...,a_{n1},...,a_{nm})  \in \mathbb{Z} ^ {nxm}$, $(b_1,...,b_n)   \in \mathbb{Z} ^ n $

On cherche $(x_1,...,x_m)  \in \mathbb{Z} ^ m$ tel que :

%$a_{11} \cdot x_1 =b_1$,$a_{11} \cdot x_1 =b_1$
% que pr systemes lineraires  \systeme{a=2, n=2}     

  %:https://tex.stackexchange.com/questions/301053/issues-with-systeme

  
%II
\subsection{Équations diophantiennes et décomposition en facteurs premiers}

%III
\subsection{Équations diophantiennes et corps de nombres quadratiques}
\textbf{Équation de Fermat pour n=3}


%IV
\subsection{Équations diophantiennes et fractions continues}





\url{https://linuxconfig.org}

%https://www.ljll.math.upmc.fr/hecht/ftp/old/InfoBase-2005-06/latex/DOC/LaTex-initiation.pdf (notes de bas de pages + marges)
\footnote{Written by Peter MOUEZA 2012}
\end{document}
